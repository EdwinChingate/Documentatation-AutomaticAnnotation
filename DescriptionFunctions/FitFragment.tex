\subsection{FitFragment}\label{FitFragment}
\subsubsection{Description}
FitFragment uses MoleculesCand to explain the difference in between two fragments. Then it compares the compare the candidates list for the two fragmets with the new list for the difference, and if the difference in between the number of atoms in the fragments fits with the molecular formula for one of the candidates for the fragments, it adds a 1 to the table D, to point that the two candidates are consistent.
\subsubsection{Arguments}
\begin{itemize}
\item \hyperref[AllPeaksAllPossibleFragments]{AllPeaksAllPossibleFragments}
\item \hyperref[AdjacencyMatDF]{AdjacencyMatDF}
\item \hyperref[PeakMass]{PeakMass1}
\item \hyperref[PeakMass]{PeakMass2}
\end{itemize}
\subsubsection{Internal variables}
\subsubsection{Used functions}
\begin{itemize}
\item \hyperref[MoleculesCand]{MoleculesCand}
\end{itemize}
\subsubsection{Output}
\begin{itemize}
\item \hyperref[AdjacencyMatDF]{AdjacencyMatDF}
\end{itemize}
\subsubsection{Usage}



