\subsection{AnnotateSpec}\label{AnnotateSpec}
\subsubsection{Description}
AnnotateSpec looks for MS2 spectra in a dataset given the exact mass (MM) and retention time (RT), then it looks for all possible ions with similar mass as the identified fragments and the differences to validate if two fragments are consistent. Finally, AnnotateSPec builds all the possible fragments networks and based on the grade it annotate the spectra.
\subsubsection{Arguments}
\begin{itemize}
\item \hyperref[DataSetName]{DataSetName}
\item \hyperref[PrecursorFragmentMass]{PrecursorFragmentMass}
\item \hyperref[RT]{RT}
\end{itemize}
\subsubsection{Parameters}
\begin{itemize}
\item \hyperref[SaveAnnotation]{SaveAnnotation}
\end{itemize}
\subsubsection{Internal variables}
\subsubsection{Used functions}
\begin{itemize}
\item \hyperref[FragSpacePos]{FragSpacePos}
\item \hyperref[SelfConsistFrag]{SelfConsistFrag}
\item \hyperref[MinEdges]{MinEdges}
\item \hyperref[FragNetIntRes]{FragNetIntRes}
\item \hyperref[IndexLists]{IndexLists}
\item \hyperref[AllNet]{AllNet}
\item \hyperref[GradeNet]{GradeNet}
\end{itemize}
\subsubsection{Output}
\begin{itemize}
\item \hyperref[AnSpec]{AnSpec}
\end{itemize}
\subsubsection{Usage}
