\subsection{FragNet}\label{FragNet}
\subsubsection{Description}
FragNet creates a list of possible networks given the possibilities for the nodes. Nodes in the network corresponds with the fragments detected, and the possibilities could be using or not the node, or all the possible ions. 
\subsubsection{Arguments}
\begin{itemize}
\item \hyperref[FragmentsNetwork]{FragmentsNetwork} 
\item \hyperref[FragmentsNetworks]{FragmentsNetworks}
\item \hyperref[FragmentPosition]{FragmentPosition}
\item \hyperref[ListofFragmentsinListofPeaks]{ListofFragmentsinListofPeaks}
\item \hyperref[PeakPosition]{PeakPosition}
\item \hyperref[PeaksNetwork]{PeaksNetwork}
\item \hyperref[StartNetworking]{StartNetworking}
\end{itemize}
\subsubsection{Internal variables}
\subsubsection{Used functions}
\begin{itemize}
\item \hyperref[FragNet]{FragNet}
\end{itemize}
\subsubsection{Output}
\begin{itemize}
\item \hyperref[FragmentsNetworks]{FragmentsNetworks}
\end{itemize}
\subsubsection{Usage}



