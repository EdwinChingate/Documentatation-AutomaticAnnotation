\subsection{NewReduceSpecFindPeaks}\label{NewReduceSpecFindPeaks}
\subsubsection{Description}

NewReduceSpecFindPeaks clusters the peaks into groups by differences in the m/z ration using ponderated statistics and the Welch test to make sure the groups are different. Each group could correspond with a fragment. It returns a list of peaks and their corresponding confidence interval.

\subsubsection{Arguments}
\begin{itemize}
\item \hyperref[RawSignals]{RawSignals}
\end{itemize}
\subsubsection{Parameters}
\begin{itemize}
\item \hyperref[ConfidenceIntervalTolerance]{ConfidenceIntervalTolerance}
\item \hyperref[MinInttobePeak]{MinInttobePeak}
\item \hyperref[MinRelIntCont]{MinRelIntCont}
\item \hyperref[MinSignalstobePeak]{MinSignalstobePeak}
\item \hyperref[NoiseTresInt]{NoiseTresInt}
\end{itemize}
\subsubsection{Internal variables}
\subsubsection{Used functions}
\begin{itemize}
\item \hyperref[PondMZStats]{PondMZStats}
\item \hyperref[WelchTest]{WelchTest}
\end{itemize}
\subsubsection{Output}
\begin{itemize}

\item \hyperref[SpectrumPeaks]{SpectrumPeaks} %when describing the variables, include a sample table
\end{itemize}

\subsubsection{Usage}

